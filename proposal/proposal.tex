\documentclass[11pt]{article}
\usepackage[backend=biber]{biblatex}
\usepackage{hyperref}
\usepackage{parskip}
\usepackage{graphicx}
\usepackage{tabularx}

\addbibresource{sources.bib}

\title{Smart Campus Humidifer System Proposal
\\ SOEN 422}
\author{Sam Anthony 40271987}
\date{\today\\Revision 1}

\begin{document}

\maketitle

\section{Problem}

Humidity is the amount of water vapor in a given amount of air.
Relative humidity (RH) is the ratio of humidity to the maximum possible humidity before condensation occurs.
Health Canada recommends a RH of 35--50\%.
However, during the winter when indoor heating is used, RH can fall below 30\%.
This can cause skin and eye irritation, and can lead to respiratory infections~\cite{healthcanada}.
The RH of campus buildings should be regulated, especially during the winter.

A solution should meet the following criteria:
\begin{enumerate}
	\item regulate indoor humidity,
	\item integrate with existing on-campus HVAC systems,
	\item be economical to install and maintain,
	\item be energy efficient, and
	\item be easily configurable.
\end{enumerate}

\section{Proposed Solution}

The proposed solution is a humidifier with a closed-loop control system.
It will be a \emph{fog type} humidifier with a high-pressure pump and atomizing nozzle that injects water droplets into the air stream.
This will minimize cost and energy compared to a \emph{steam type} humidifier because no heating element is required.
According to Armstrong, ``pressure fog is a system that is perfect for applications requiring high humidification output with minimal energy consumption" \cite{armstrong}.

The water nozzle can easily be installed in the existing HVAC ductwork.

Inhabitants of the building will be able to adjust the desired humidity level from their smart phones.

\section{Initial Design}

A microcontroller equiped with a humidity sensor and a WiFi and Bluetooth module will be installed in each room of the building.
The humidity of each room will be sampled periodically and sent to a remote server via the WiFi network.
Users will connect from their phone and set the target humidity via bluetooth.
The target humidity will also be sent to the server over WiFi.

The server will store the target humidity and will maintain a log of humidity measurements for each room.
With this data it can calculate a running average humidity for the building.

Another microcontroller, also WiFi-capable, will be installed in the central HVAC room of the building.
It will connect to the server via WiFi and retreive the target humidity and current average humidity of the building.
A PID algorithm will be used to determine the correct duty cycle of the humidifier in order to regulate the humidity to the target level.
To control the humidifier, the microcontroller will use PWM to actuate a solenoid valve between the water pump and the injection nozzle.

The HVAC room microcontroller will post the current duty cycle to the server for monitoring.
The server will provide a graphical interface that displays historical humidity data for each room and for the entire building, as well as the current duty cycle of the humidifier and the  target humidity.

\fbox{\includegraphics[width=\textwidth]{"diagram.png"}}

\section{Hardware}

\begin{tabularx}{\textwidth}{r X X}
	\hline
	Quantity & Device & Notes \\
	\hline
	1/room & DHT11 humidity sensor \\
	1/room + 1 & ESP32 & one per room, and one for the HVAC room \\
	1 & High-pressure water pump \\
	1 & Solenoid valve \\
	1 & Atomizing nozzle \\
	 - & Water lines \\
	\hline
\end{tabularx}

\printbibliography

\end{document}